\documentclass[11pt]{article}

    \usepackage[breakable]{tcolorbox}
    \usepackage{parskip} % Stop auto-indenting (to mimic markdown behaviour)
    
    \usepackage{iftex}
    \ifPDFTeX
    	\usepackage[T1]{fontenc}
    	\usepackage{mathpazo}
    \else
    	\usepackage{fontspec}
    \fi

    % Basic figure setup, for now with no caption control since it's done
    % automatically by Pandoc (which extracts ![](path) syntax from Markdown).
    \usepackage{graphicx}
    % Maintain compatibility with old templates. Remove in nbconvert 6.0
    \let\Oldincludegraphics\includegraphics
    % Ensure that by default, figures have no caption (until we provide a
    % proper Figure object with a Caption API and a way to capture that
    % in the conversion process - todo).
    \usepackage{caption}
    \DeclareCaptionFormat{nocaption}{}
    \captionsetup{format=nocaption,aboveskip=0pt,belowskip=0pt}

    \usepackage{float}
    \floatplacement{figure}{H} % forces figures to be placed at the correct location
    \usepackage{xcolor} % Allow colors to be defined
    \usepackage{enumerate} % Needed for markdown enumerations to work
    \usepackage{geometry} % Used to adjust the document margins
    \usepackage{amsmath} % Equations
    \usepackage{amssymb} % Equations
    \usepackage{textcomp} % defines textquotesingle
    % Hack from http://tex.stackexchange.com/a/47451/13684:
    \AtBeginDocument{%
        \def\PYZsq{\textquotesingle}% Upright quotes in Pygmentized code
    }
    \usepackage{upquote} % Upright quotes for verbatim code
    \usepackage{eurosym} % defines \euro
    \usepackage[mathletters]{ucs} % Extended unicode (utf-8) support
    \usepackage{fancyvrb} % verbatim replacement that allows latex
    \usepackage{grffile} % extends the file name processing of package graphics 
                         % to support a larger range
    \makeatletter % fix for old versions of grffile with XeLaTeX
    \@ifpackagelater{grffile}{2019/11/01}
    {
      % Do nothing on new versions
    }
    {
      \def\Gread@@xetex#1{%
        \IfFileExists{"\Gin@base".bb}%
        {\Gread@eps{\Gin@base.bb}}%
        {\Gread@@xetex@aux#1}%
      }
    }
    \makeatother
    \usepackage[Export]{adjustbox} % Used to constrain images to a maximum size
    \adjustboxset{max size={0.9\linewidth}{0.9\paperheight}}

    % The hyperref package gives us a pdf with properly built
    % internal navigation ('pdf bookmarks' for the table of contents,
    % internal cross-reference links, web links for URLs, etc.)
    \usepackage{hyperref}
    % The default LaTeX title has an obnoxious amount of whitespace. By default,
    % titling removes some of it. It also provides customization options.
    \usepackage{titling}
    \usepackage{longtable} % longtable support required by pandoc >1.10
    \usepackage{booktabs}  % table support for pandoc > 1.12.2
    \usepackage[inline]{enumitem} % IRkernel/repr support (it uses the enumerate* environment)
    \usepackage[normalem]{ulem} % ulem is needed to support strikethroughs (\sout)
                                % normalem makes italics be italics, not underlines
    \usepackage{mathrsfs}
    

    
    % Colors for the hyperref package
    \definecolor{urlcolor}{rgb}{0,.145,.698}
    \definecolor{linkcolor}{rgb}{.71,0.21,0.01}
    \definecolor{citecolor}{rgb}{.12,.54,.11}

    % ANSI colors
    \definecolor{ansi-black}{HTML}{3E424D}
    \definecolor{ansi-black-intense}{HTML}{282C36}
    \definecolor{ansi-red}{HTML}{E75C58}
    \definecolor{ansi-red-intense}{HTML}{B22B31}
    \definecolor{ansi-green}{HTML}{00A250}
    \definecolor{ansi-green-intense}{HTML}{007427}
    \definecolor{ansi-yellow}{HTML}{DDB62B}
    \definecolor{ansi-yellow-intense}{HTML}{B27D12}
    \definecolor{ansi-blue}{HTML}{208FFB}
    \definecolor{ansi-blue-intense}{HTML}{0065CA}
    \definecolor{ansi-magenta}{HTML}{D160C4}
    \definecolor{ansi-magenta-intense}{HTML}{A03196}
    \definecolor{ansi-cyan}{HTML}{60C6C8}
    \definecolor{ansi-cyan-intense}{HTML}{258F8F}
    \definecolor{ansi-white}{HTML}{C5C1B4}
    \definecolor{ansi-white-intense}{HTML}{A1A6B2}
    \definecolor{ansi-default-inverse-fg}{HTML}{FFFFFF}
    \definecolor{ansi-default-inverse-bg}{HTML}{000000}

    % common color for the border for error outputs.
    \definecolor{outerrorbackground}{HTML}{FFDFDF}

    % commands and environments needed by pandoc snippets
    % extracted from the output of `pandoc -s`
    \providecommand{\tightlist}{%
      \setlength{\itemsep}{0pt}\setlength{\parskip}{0pt}}
    \DefineVerbatimEnvironment{Highlighting}{Verbatim}{commandchars=\\\{\}}
    % Add ',fontsize=\small' for more characters per line
    \newenvironment{Shaded}{}{}
    \newcommand{\KeywordTok}[1]{\textcolor[rgb]{0.00,0.44,0.13}{\textbf{{#1}}}}
    \newcommand{\DataTypeTok}[1]{\textcolor[rgb]{0.56,0.13,0.00}{{#1}}}
    \newcommand{\DecValTok}[1]{\textcolor[rgb]{0.25,0.63,0.44}{{#1}}}
    \newcommand{\BaseNTok}[1]{\textcolor[rgb]{0.25,0.63,0.44}{{#1}}}
    \newcommand{\FloatTok}[1]{\textcolor[rgb]{0.25,0.63,0.44}{{#1}}}
    \newcommand{\CharTok}[1]{\textcolor[rgb]{0.25,0.44,0.63}{{#1}}}
    \newcommand{\StringTok}[1]{\textcolor[rgb]{0.25,0.44,0.63}{{#1}}}
    \newcommand{\CommentTok}[1]{\textcolor[rgb]{0.38,0.63,0.69}{\textit{{#1}}}}
    \newcommand{\OtherTok}[1]{\textcolor[rgb]{0.00,0.44,0.13}{{#1}}}
    \newcommand{\AlertTok}[1]{\textcolor[rgb]{1.00,0.00,0.00}{\textbf{{#1}}}}
    \newcommand{\FunctionTok}[1]{\textcolor[rgb]{0.02,0.16,0.49}{{#1}}}
    \newcommand{\RegionMarkerTok}[1]{{#1}}
    \newcommand{\ErrorTok}[1]{\textcolor[rgb]{1.00,0.00,0.00}{\textbf{{#1}}}}
    \newcommand{\NormalTok}[1]{{#1}}
    
    % Additional commands for more recent versions of Pandoc
    \newcommand{\ConstantTok}[1]{\textcolor[rgb]{0.53,0.00,0.00}{{#1}}}
    \newcommand{\SpecialCharTok}[1]{\textcolor[rgb]{0.25,0.44,0.63}{{#1}}}
    \newcommand{\VerbatimStringTok}[1]{\textcolor[rgb]{0.25,0.44,0.63}{{#1}}}
    \newcommand{\SpecialStringTok}[1]{\textcolor[rgb]{0.73,0.40,0.53}{{#1}}}
    \newcommand{\ImportTok}[1]{{#1}}
    \newcommand{\DocumentationTok}[1]{\textcolor[rgb]{0.73,0.13,0.13}{\textit{{#1}}}}
    \newcommand{\AnnotationTok}[1]{\textcolor[rgb]{0.38,0.63,0.69}{\textbf{\textit{{#1}}}}}
    \newcommand{\CommentVarTok}[1]{\textcolor[rgb]{0.38,0.63,0.69}{\textbf{\textit{{#1}}}}}
    \newcommand{\VariableTok}[1]{\textcolor[rgb]{0.10,0.09,0.49}{{#1}}}
    \newcommand{\ControlFlowTok}[1]{\textcolor[rgb]{0.00,0.44,0.13}{\textbf{{#1}}}}
    \newcommand{\OperatorTok}[1]{\textcolor[rgb]{0.40,0.40,0.40}{{#1}}}
    \newcommand{\BuiltInTok}[1]{{#1}}
    \newcommand{\ExtensionTok}[1]{{#1}}
    \newcommand{\PreprocessorTok}[1]{\textcolor[rgb]{0.74,0.48,0.00}{{#1}}}
    \newcommand{\AttributeTok}[1]{\textcolor[rgb]{0.49,0.56,0.16}{{#1}}}
    \newcommand{\InformationTok}[1]{\textcolor[rgb]{0.38,0.63,0.69}{\textbf{\textit{{#1}}}}}
    \newcommand{\WarningTok}[1]{\textcolor[rgb]{0.38,0.63,0.69}{\textbf{\textit{{#1}}}}}
    
    
    % Define a nice break command that doesn't care if a line doesn't already
    % exist.
    \def\br{\hspace*{\fill} \\* }
    % Math Jax compatibility definitions
    \def\gt{>}
    \def\lt{<}
    \let\Oldtex\TeX
    \let\Oldlatex\LaTeX
    \renewcommand{\TeX}{\textrm{\Oldtex}}
    \renewcommand{\LaTeX}{\textrm{\Oldlatex}}
    % Document parameters
    % Document title
    \title{MPI - Paweł Kruczkiewicz}
    
    
    
    
    
% Pygments definitions
\makeatletter
\def\PY@reset{\let\PY@it=\relax \let\PY@bf=\relax%
    \let\PY@ul=\relax \let\PY@tc=\relax%
    \let\PY@bc=\relax \let\PY@ff=\relax}
\def\PY@tok#1{\csname PY@tok@#1\endcsname}
\def\PY@toks#1+{\ifx\relax#1\empty\else%
    \PY@tok{#1}\expandafter\PY@toks\fi}
\def\PY@do#1{\PY@bc{\PY@tc{\PY@ul{%
    \PY@it{\PY@bf{\PY@ff{#1}}}}}}}
\def\PY#1#2{\PY@reset\PY@toks#1+\relax+\PY@do{#2}}

\@namedef{PY@tok@w}{\def\PY@tc##1{\textcolor[rgb]{0.73,0.73,0.73}{##1}}}
\@namedef{PY@tok@c}{\let\PY@it=\textit\def\PY@tc##1{\textcolor[rgb]{0.24,0.48,0.48}{##1}}}
\@namedef{PY@tok@cp}{\def\PY@tc##1{\textcolor[rgb]{0.61,0.40,0.00}{##1}}}
\@namedef{PY@tok@k}{\let\PY@bf=\textbf\def\PY@tc##1{\textcolor[rgb]{0.00,0.50,0.00}{##1}}}
\@namedef{PY@tok@kp}{\def\PY@tc##1{\textcolor[rgb]{0.00,0.50,0.00}{##1}}}
\@namedef{PY@tok@kt}{\def\PY@tc##1{\textcolor[rgb]{0.69,0.00,0.25}{##1}}}
\@namedef{PY@tok@o}{\def\PY@tc##1{\textcolor[rgb]{0.40,0.40,0.40}{##1}}}
\@namedef{PY@tok@ow}{\let\PY@bf=\textbf\def\PY@tc##1{\textcolor[rgb]{0.67,0.13,1.00}{##1}}}
\@namedef{PY@tok@nb}{\def\PY@tc##1{\textcolor[rgb]{0.00,0.50,0.00}{##1}}}
\@namedef{PY@tok@nf}{\def\PY@tc##1{\textcolor[rgb]{0.00,0.00,1.00}{##1}}}
\@namedef{PY@tok@nc}{\let\PY@bf=\textbf\def\PY@tc##1{\textcolor[rgb]{0.00,0.00,1.00}{##1}}}
\@namedef{PY@tok@nn}{\let\PY@bf=\textbf\def\PY@tc##1{\textcolor[rgb]{0.00,0.00,1.00}{##1}}}
\@namedef{PY@tok@ne}{\let\PY@bf=\textbf\def\PY@tc##1{\textcolor[rgb]{0.80,0.25,0.22}{##1}}}
\@namedef{PY@tok@nv}{\def\PY@tc##1{\textcolor[rgb]{0.10,0.09,0.49}{##1}}}
\@namedef{PY@tok@no}{\def\PY@tc##1{\textcolor[rgb]{0.53,0.00,0.00}{##1}}}
\@namedef{PY@tok@nl}{\def\PY@tc##1{\textcolor[rgb]{0.46,0.46,0.00}{##1}}}
\@namedef{PY@tok@ni}{\let\PY@bf=\textbf\def\PY@tc##1{\textcolor[rgb]{0.44,0.44,0.44}{##1}}}
\@namedef{PY@tok@na}{\def\PY@tc##1{\textcolor[rgb]{0.41,0.47,0.13}{##1}}}
\@namedef{PY@tok@nt}{\let\PY@bf=\textbf\def\PY@tc##1{\textcolor[rgb]{0.00,0.50,0.00}{##1}}}
\@namedef{PY@tok@nd}{\def\PY@tc##1{\textcolor[rgb]{0.67,0.13,1.00}{##1}}}
\@namedef{PY@tok@s}{\def\PY@tc##1{\textcolor[rgb]{0.73,0.13,0.13}{##1}}}
\@namedef{PY@tok@sd}{\let\PY@it=\textit\def\PY@tc##1{\textcolor[rgb]{0.73,0.13,0.13}{##1}}}
\@namedef{PY@tok@si}{\let\PY@bf=\textbf\def\PY@tc##1{\textcolor[rgb]{0.64,0.35,0.47}{##1}}}
\@namedef{PY@tok@se}{\let\PY@bf=\textbf\def\PY@tc##1{\textcolor[rgb]{0.67,0.36,0.12}{##1}}}
\@namedef{PY@tok@sr}{\def\PY@tc##1{\textcolor[rgb]{0.64,0.35,0.47}{##1}}}
\@namedef{PY@tok@ss}{\def\PY@tc##1{\textcolor[rgb]{0.10,0.09,0.49}{##1}}}
\@namedef{PY@tok@sx}{\def\PY@tc##1{\textcolor[rgb]{0.00,0.50,0.00}{##1}}}
\@namedef{PY@tok@m}{\def\PY@tc##1{\textcolor[rgb]{0.40,0.40,0.40}{##1}}}
\@namedef{PY@tok@gh}{\let\PY@bf=\textbf\def\PY@tc##1{\textcolor[rgb]{0.00,0.00,0.50}{##1}}}
\@namedef{PY@tok@gu}{\let\PY@bf=\textbf\def\PY@tc##1{\textcolor[rgb]{0.50,0.00,0.50}{##1}}}
\@namedef{PY@tok@gd}{\def\PY@tc##1{\textcolor[rgb]{0.63,0.00,0.00}{##1}}}
\@namedef{PY@tok@gi}{\def\PY@tc##1{\textcolor[rgb]{0.00,0.52,0.00}{##1}}}
\@namedef{PY@tok@gr}{\def\PY@tc##1{\textcolor[rgb]{0.89,0.00,0.00}{##1}}}
\@namedef{PY@tok@ge}{\let\PY@it=\textit}
\@namedef{PY@tok@gs}{\let\PY@bf=\textbf}
\@namedef{PY@tok@gp}{\let\PY@bf=\textbf\def\PY@tc##1{\textcolor[rgb]{0.00,0.00,0.50}{##1}}}
\@namedef{PY@tok@go}{\def\PY@tc##1{\textcolor[rgb]{0.44,0.44,0.44}{##1}}}
\@namedef{PY@tok@gt}{\def\PY@tc##1{\textcolor[rgb]{0.00,0.27,0.87}{##1}}}
\@namedef{PY@tok@err}{\def\PY@bc##1{{\setlength{\fboxsep}{\string -\fboxrule}\fcolorbox[rgb]{1.00,0.00,0.00}{1,1,1}{\strut ##1}}}}
\@namedef{PY@tok@kc}{\let\PY@bf=\textbf\def\PY@tc##1{\textcolor[rgb]{0.00,0.50,0.00}{##1}}}
\@namedef{PY@tok@kd}{\let\PY@bf=\textbf\def\PY@tc##1{\textcolor[rgb]{0.00,0.50,0.00}{##1}}}
\@namedef{PY@tok@kn}{\let\PY@bf=\textbf\def\PY@tc##1{\textcolor[rgb]{0.00,0.50,0.00}{##1}}}
\@namedef{PY@tok@kr}{\let\PY@bf=\textbf\def\PY@tc##1{\textcolor[rgb]{0.00,0.50,0.00}{##1}}}
\@namedef{PY@tok@bp}{\def\PY@tc##1{\textcolor[rgb]{0.00,0.50,0.00}{##1}}}
\@namedef{PY@tok@fm}{\def\PY@tc##1{\textcolor[rgb]{0.00,0.00,1.00}{##1}}}
\@namedef{PY@tok@vc}{\def\PY@tc##1{\textcolor[rgb]{0.10,0.09,0.49}{##1}}}
\@namedef{PY@tok@vg}{\def\PY@tc##1{\textcolor[rgb]{0.10,0.09,0.49}{##1}}}
\@namedef{PY@tok@vi}{\def\PY@tc##1{\textcolor[rgb]{0.10,0.09,0.49}{##1}}}
\@namedef{PY@tok@vm}{\def\PY@tc##1{\textcolor[rgb]{0.10,0.09,0.49}{##1}}}
\@namedef{PY@tok@sa}{\def\PY@tc##1{\textcolor[rgb]{0.73,0.13,0.13}{##1}}}
\@namedef{PY@tok@sb}{\def\PY@tc##1{\textcolor[rgb]{0.73,0.13,0.13}{##1}}}
\@namedef{PY@tok@sc}{\def\PY@tc##1{\textcolor[rgb]{0.73,0.13,0.13}{##1}}}
\@namedef{PY@tok@dl}{\def\PY@tc##1{\textcolor[rgb]{0.73,0.13,0.13}{##1}}}
\@namedef{PY@tok@s2}{\def\PY@tc##1{\textcolor[rgb]{0.73,0.13,0.13}{##1}}}
\@namedef{PY@tok@sh}{\def\PY@tc##1{\textcolor[rgb]{0.73,0.13,0.13}{##1}}}
\@namedef{PY@tok@s1}{\def\PY@tc##1{\textcolor[rgb]{0.73,0.13,0.13}{##1}}}
\@namedef{PY@tok@mb}{\def\PY@tc##1{\textcolor[rgb]{0.40,0.40,0.40}{##1}}}
\@namedef{PY@tok@mf}{\def\PY@tc##1{\textcolor[rgb]{0.40,0.40,0.40}{##1}}}
\@namedef{PY@tok@mh}{\def\PY@tc##1{\textcolor[rgb]{0.40,0.40,0.40}{##1}}}
\@namedef{PY@tok@mi}{\def\PY@tc##1{\textcolor[rgb]{0.40,0.40,0.40}{##1}}}
\@namedef{PY@tok@il}{\def\PY@tc##1{\textcolor[rgb]{0.40,0.40,0.40}{##1}}}
\@namedef{PY@tok@mo}{\def\PY@tc##1{\textcolor[rgb]{0.40,0.40,0.40}{##1}}}
\@namedef{PY@tok@ch}{\let\PY@it=\textit\def\PY@tc##1{\textcolor[rgb]{0.24,0.48,0.48}{##1}}}
\@namedef{PY@tok@cm}{\let\PY@it=\textit\def\PY@tc##1{\textcolor[rgb]{0.24,0.48,0.48}{##1}}}
\@namedef{PY@tok@cpf}{\let\PY@it=\textit\def\PY@tc##1{\textcolor[rgb]{0.24,0.48,0.48}{##1}}}
\@namedef{PY@tok@c1}{\let\PY@it=\textit\def\PY@tc##1{\textcolor[rgb]{0.24,0.48,0.48}{##1}}}
\@namedef{PY@tok@cs}{\let\PY@it=\textit\def\PY@tc##1{\textcolor[rgb]{0.24,0.48,0.48}{##1}}}

\def\PYZbs{\char`\\}
\def\PYZus{\char`\_}
\def\PYZob{\char`\{}
\def\PYZcb{\char`\}}
\def\PYZca{\char`\^}
\def\PYZam{\char`\&}
\def\PYZlt{\char`\<}
\def\PYZgt{\char`\>}
\def\PYZsh{\char`\#}
\def\PYZpc{\char`\%}
\def\PYZdl{\char`\$}
\def\PYZhy{\char`\-}
\def\PYZsq{\char`\'}
\def\PYZdq{\char`\"}
\def\PYZti{\char`\~}
% for compatibility with earlier versions
\def\PYZat{@}
\def\PYZlb{[}
\def\PYZrb{]}
\makeatother


    % For linebreaks inside Verbatim environment from package fancyvrb. 
    \makeatletter
        \newbox\Wrappedcontinuationbox 
        \newbox\Wrappedvisiblespacebox 
        \newcommand*\Wrappedvisiblespace {\textcolor{red}{\textvisiblespace}} 
        \newcommand*\Wrappedcontinuationsymbol {\textcolor{red}{\llap{\tiny$\m@th\hookrightarrow$}}} 
        \newcommand*\Wrappedcontinuationindent {3ex } 
        \newcommand*\Wrappedafterbreak {\kern\Wrappedcontinuationindent\copy\Wrappedcontinuationbox} 
        % Take advantage of the already applied Pygments mark-up to insert 
        % potential linebreaks for TeX processing. 
        %        {, <, #, %, $, ' and ": go to next line. 
        %        _, }, ^, &, >, - and ~: stay at end of broken line. 
        % Use of \textquotesingle for straight quote. 
        \newcommand*\Wrappedbreaksatspecials {% 
            \def\PYGZus{\discretionary{\char`\_}{\Wrappedafterbreak}{\char`\_}}% 
            \def\PYGZob{\discretionary{}{\Wrappedafterbreak\char`\{}{\char`\{}}% 
            \def\PYGZcb{\discretionary{\char`\}}{\Wrappedafterbreak}{\char`\}}}% 
            \def\PYGZca{\discretionary{\char`\^}{\Wrappedafterbreak}{\char`\^}}% 
            \def\PYGZam{\discretionary{\char`\&}{\Wrappedafterbreak}{\char`\&}}% 
            \def\PYGZlt{\discretionary{}{\Wrappedafterbreak\char`\<}{\char`\<}}% 
            \def\PYGZgt{\discretionary{\char`\>}{\Wrappedafterbreak}{\char`\>}}% 
            \def\PYGZsh{\discretionary{}{\Wrappedafterbreak\char`\#}{\char`\#}}% 
            \def\PYGZpc{\discretionary{}{\Wrappedafterbreak\char`\%}{\char`\%}}% 
            \def\PYGZdl{\discretionary{}{\Wrappedafterbreak\char`\$}{\char`\$}}% 
            \def\PYGZhy{\discretionary{\char`\-}{\Wrappedafterbreak}{\char`\-}}% 
            \def\PYGZsq{\discretionary{}{\Wrappedafterbreak\textquotesingle}{\textquotesingle}}% 
            \def\PYGZdq{\discretionary{}{\Wrappedafterbreak\char`\"}{\char`\"}}% 
            \def\PYGZti{\discretionary{\char`\~}{\Wrappedafterbreak}{\char`\~}}% 
        } 
        % Some characters . , ; ? ! / are not pygmentized. 
        % This macro makes them "active" and they will insert potential linebreaks 
        \newcommand*\Wrappedbreaksatpunct {% 
            \lccode`\~`\.\lowercase{\def~}{\discretionary{\hbox{\char`\.}}{\Wrappedafterbreak}{\hbox{\char`\.}}}% 
            \lccode`\~`\,\lowercase{\def~}{\discretionary{\hbox{\char`\,}}{\Wrappedafterbreak}{\hbox{\char`\,}}}% 
            \lccode`\~`\;\lowercase{\def~}{\discretionary{\hbox{\char`\;}}{\Wrappedafterbreak}{\hbox{\char`\;}}}% 
            \lccode`\~`\:\lowercase{\def~}{\discretionary{\hbox{\char`\:}}{\Wrappedafterbreak}{\hbox{\char`\:}}}% 
            \lccode`\~`\?\lowercase{\def~}{\discretionary{\hbox{\char`\?}}{\Wrappedafterbreak}{\hbox{\char`\?}}}% 
            \lccode`\~`\!\lowercase{\def~}{\discretionary{\hbox{\char`\!}}{\Wrappedafterbreak}{\hbox{\char`\!}}}% 
            \lccode`\~`\/\lowercase{\def~}{\discretionary{\hbox{\char`\/}}{\Wrappedafterbreak}{\hbox{\char`\/}}}% 
            \catcode`\.\active
            \catcode`\,\active 
            \catcode`\;\active
            \catcode`\:\active
            \catcode`\?\active
            \catcode`\!\active
            \catcode`\/\active 
            \lccode`\~`\~ 	
        }
    \makeatother

    \let\OriginalVerbatim=\Verbatim
    \makeatletter
    \renewcommand{\Verbatim}[1][1]{%
        %\parskip\z@skip
        \sbox\Wrappedcontinuationbox {\Wrappedcontinuationsymbol}%
        \sbox\Wrappedvisiblespacebox {\FV@SetupFont\Wrappedvisiblespace}%
        \def\FancyVerbFormatLine ##1{\hsize\linewidth
            \vtop{\raggedright\hyphenpenalty\z@\exhyphenpenalty\z@
                \doublehyphendemerits\z@\finalhyphendemerits\z@
                \strut ##1\strut}%
        }%
        % If the linebreak is at a space, the latter will be displayed as visible
        % space at end of first line, and a continuation symbol starts next line.
        % Stretch/shrink are however usually zero for typewriter font.
        \def\FV@Space {%
            \nobreak\hskip\z@ plus\fontdimen3\font minus\fontdimen4\font
            \discretionary{\copy\Wrappedvisiblespacebox}{\Wrappedafterbreak}
            {\kern\fontdimen2\font}%
        }%
        
        % Allow breaks at special characters using \PYG... macros.
        \Wrappedbreaksatspecials
        % Breaks at punctuation characters . , ; ? ! and / need catcode=\active 	
        \OriginalVerbatim[#1,codes*=\Wrappedbreaksatpunct]%
    }
    \makeatother

    % Exact colors from NB
    \definecolor{incolor}{HTML}{303F9F}
    \definecolor{outcolor}{HTML}{D84315}
    \definecolor{cellborder}{HTML}{CFCFCF}
    \definecolor{cellbackground}{HTML}{F7F7F7}
    
    % prompt
    \makeatletter
    \newcommand{\boxspacing}{\kern\kvtcb@left@rule\kern\kvtcb@boxsep}
    \makeatother
    \newcommand{\prompt}[4]{
        {\ttfamily\llap{{\color{#2}[#3]:\hspace{3pt}#4}}\vspace{-\baselineskip}}
    }
    

    
    % Prevent overflowing lines due to hard-to-break entities
    \sloppy 
    % Setup hyperref package
    \hypersetup{
      breaklinks=true,  % so long urls are correctly broken across lines
      colorlinks=true,
      urlcolor=urlcolor,
      linkcolor=linkcolor,
      citecolor=citecolor,
      }
    % Slightly bigger margins than the latex defaults
    
    \geometry{verbose,tmargin=1in,bmargin=1in,lmargin=1in,rmargin=1in}
    
    

\begin{document}
    
    \maketitle
    
    

    
    \hypertarget{metody-programowania-ruxf3wnolegux142ego}{%
\section{Metody Programowania
Równoległego}\label{metody-programowania-ruxf3wnolegux142ego}}

\emph{Temat: Message Passing Interface}

Wykonał: \textbf{Paweł Kruczkiewicz}

            \begin{tcolorbox}[breakable, size=fbox, boxrule=.5pt, pad at break*=1mm, opacityfill=0]
\begin{Verbatim}[commandchars=\\\{\}]
<Figure size 640x480 with 0 Axes>
\end{Verbatim}
\end{tcolorbox}
        
    
    \begin{Verbatim}[commandchars=\\\{\}]
<Figure size 640x480 with 0 Axes>
    \end{Verbatim}

    
    \hypertarget{kod-programu}{%
\subsection{Kod Programu}\label{kod-programu}}

Do wykonania pomiarów posłużono się poniżej przedstawionym programem
napisanym w języku C:

\begin{Shaded}
\begin{Highlighting}[]
\PreprocessorTok{\#include }\ImportTok{\textless{}stdio.h\textgreater{}}
\PreprocessorTok{\#include }\ImportTok{\textless{}stdlib.h\textgreater{}}
\PreprocessorTok{\#include }\ImportTok{\textless{}time.h\textgreater{}}
\PreprocessorTok{\#include }\ImportTok{\textless{}mpi.h\textgreater{}}

\PreprocessorTok{\#define N 1001}
\PreprocessorTok{\#define DEFAULT\_TAG 0}
\PreprocessorTok{\#define TEST\_CASES\_NUM 6}
\PreprocessorTok{\#define CSV\_NAME "results/normal\_1\_node.csv"}

\DataTypeTok{void}\NormalTok{ init\_mpi}\OperatorTok{(}\DataTypeTok{int}\OperatorTok{*}\NormalTok{ argc}\OperatorTok{,} \DataTypeTok{char}\OperatorTok{**}\NormalTok{ argv}\OperatorTok{[],} \DataTypeTok{int}\OperatorTok{*}\NormalTok{ rank}\OperatorTok{,} \DataTypeTok{int}\OperatorTok{*}\NormalTok{ size}\OperatorTok{)\{}
\NormalTok{   MPI\_Init }\OperatorTok{(}\NormalTok{argc}\OperatorTok{,}\NormalTok{ argv}\OperatorTok{);}  \CommentTok{/* starts MPI */}
\NormalTok{  MPI\_Comm\_rank }\OperatorTok{(}\NormalTok{MPI\_COMM\_WORLD}\OperatorTok{,}\NormalTok{ rank}\OperatorTok{);}  \CommentTok{/* get current process id */}
\NormalTok{  MPI\_Comm\_size }\OperatorTok{(}\NormalTok{MPI\_COMM\_WORLD}\OperatorTok{,}\NormalTok{ size}\OperatorTok{);}  \CommentTok{/* get number of processes */}
\OperatorTok{\}}

\DataTypeTok{void}\NormalTok{ my\_MPI\_send}\OperatorTok{(}\DataTypeTok{int}\OperatorTok{*}\NormalTok{ number\_buf}\OperatorTok{,} \DataTypeTok{int}\NormalTok{ number\_amount}\OperatorTok{,} \DataTypeTok{int}\NormalTok{ receiver}\OperatorTok{)} \OperatorTok{\{}
  \PreprocessorTok{\# ifdef BUFFERED}
        \DataTypeTok{int}\NormalTok{ buffer\_attached\_size }\OperatorTok{=}\NormalTok{ MPI\_BSEND\_OVERHEAD }\OperatorTok{+}\NormalTok{  number\_amount}\OperatorTok{*}\KeywordTok{sizeof}\OperatorTok{(}\DataTypeTok{int}\OperatorTok{);}
        \DataTypeTok{char}\OperatorTok{*}\NormalTok{ buffer\_attached }\OperatorTok{=} \OperatorTok{(}\DataTypeTok{char}\OperatorTok{*)}\NormalTok{ malloc}\OperatorTok{(}\NormalTok{buffer\_attached\_size}\OperatorTok{);}
\NormalTok{        MPI\_Buffer\_attach}\OperatorTok{(}\NormalTok{buffer\_attached}\OperatorTok{,}\NormalTok{ buffer\_attached\_size}\OperatorTok{);}

\NormalTok{        MPI\_Bsend}\OperatorTok{(}\NormalTok{number\_buf}\OperatorTok{,}\NormalTok{ number\_amount}\OperatorTok{,}\NormalTok{ MPI\_INT}\OperatorTok{,}\NormalTok{ receiver}\OperatorTok{,}\NormalTok{ DEFAULT\_TAG}\OperatorTok{,}\NormalTok{ MPI\_COMM\_WORLD}\OperatorTok{);}
        
\NormalTok{        MPI\_Buffer\_detach}\OperatorTok{(\&}\NormalTok{buffer\_attached}\OperatorTok{,} \OperatorTok{\&}\NormalTok{buffer\_attached\_size}\OperatorTok{);}
\NormalTok{        free}\OperatorTok{(}\NormalTok{buffer\_attached}\OperatorTok{);} 
  \PreprocessorTok{\# else}
\NormalTok{      MPI\_Send}\OperatorTok{(}\NormalTok{number\_buf}\OperatorTok{,}\NormalTok{ number\_amount}\OperatorTok{,}\NormalTok{ MPI\_INT}\OperatorTok{,}\NormalTok{ receiver}\OperatorTok{,}\NormalTok{ DEFAULT\_TAG}\OperatorTok{,}\NormalTok{ MPI\_COMM\_WORLD}\OperatorTok{);}
  \PreprocessorTok{\# endif }\CommentTok{/* BUFFERED */}

\OperatorTok{\}}


\DataTypeTok{void}\NormalTok{ my\_MPI\_receive}\OperatorTok{(}\DataTypeTok{int}\OperatorTok{*}\NormalTok{ number\_buf}\OperatorTok{,} \DataTypeTok{int}\NormalTok{ number\_amount}\OperatorTok{,} \DataTypeTok{int}\NormalTok{ sender}\OperatorTok{)\{}
\NormalTok{  MPI\_Recv}\OperatorTok{(}\NormalTok{number\_buf}\OperatorTok{,}\NormalTok{ number\_amount}\OperatorTok{,}\NormalTok{ MPI\_INT}\OperatorTok{,}\NormalTok{ sender}\OperatorTok{,}\NormalTok{ DEFAULT\_TAG}\OperatorTok{,}\NormalTok{ MPI\_COMM\_WORLD}\OperatorTok{,}\NormalTok{ MPI\_STATUS\_IGNORE}\OperatorTok{);}
\OperatorTok{\}}

\DataTypeTok{void}\NormalTok{ MPI\_one\_ping\_pong}\OperatorTok{(}\DataTypeTok{int}\NormalTok{ rank}\OperatorTok{,} \DataTypeTok{int}\OperatorTok{*}\NormalTok{ number\_buf}\OperatorTok{,} \DataTypeTok{int}\NormalTok{ number\_amount}\OperatorTok{)} \OperatorTok{\{}
  \ControlFlowTok{if} \OperatorTok{(}\NormalTok{rank }\OperatorTok{==} \DecValTok{0}\OperatorTok{)} \OperatorTok{\{}
\NormalTok{    my\_MPI\_send}\OperatorTok{(}\NormalTok{number\_buf}\OperatorTok{,}\NormalTok{ number\_amount}\OperatorTok{,} \DecValTok{1}\OperatorTok{);}
\NormalTok{        my\_MPI\_receive}\OperatorTok{(}\NormalTok{number\_buf}\OperatorTok{,}\NormalTok{ number\_amount}\OperatorTok{,} \DecValTok{1}\OperatorTok{);}
  \OperatorTok{\}} \ControlFlowTok{else} \ControlFlowTok{if} \OperatorTok{(}\NormalTok{rank }\OperatorTok{==} \DecValTok{1}\OperatorTok{)} \OperatorTok{\{}
\NormalTok{        my\_MPI\_receive}\OperatorTok{(}\NormalTok{number\_buf}\OperatorTok{,}\NormalTok{ number\_amount}\OperatorTok{,} \DecValTok{0}\OperatorTok{);} 
\NormalTok{    my\_MPI\_send}\OperatorTok{(}\NormalTok{number\_buf}\OperatorTok{,}\NormalTok{ number\_amount}\OperatorTok{,} \DecValTok{0}\OperatorTok{);}
  \OperatorTok{\}}
\OperatorTok{\}}

\DataTypeTok{int}\OperatorTok{*}\NormalTok{ init\_buf}\OperatorTok{(}\DataTypeTok{int}\NormalTok{ n}\OperatorTok{)\{}
  \DataTypeTok{int}\OperatorTok{*}\NormalTok{ buf }\OperatorTok{=} \OperatorTok{(}\DataTypeTok{int}\OperatorTok{*)}\NormalTok{ malloc}\OperatorTok{(}\KeywordTok{sizeof}\OperatorTok{(}\DataTypeTok{int}\OperatorTok{)} \OperatorTok{*}\NormalTok{ n}\OperatorTok{);}
  \DataTypeTok{int}\NormalTok{ i}\OperatorTok{;}
  \ControlFlowTok{for} \OperatorTok{(}\NormalTok{i}\OperatorTok{=}\DecValTok{0}\OperatorTok{;}\NormalTok{ i }\OperatorTok{\textless{}}\NormalTok{ n}\OperatorTok{;}\NormalTok{ i}\OperatorTok{++)\{}
\NormalTok{     buf}\OperatorTok{[}\NormalTok{i}\OperatorTok{]} \OperatorTok{=} \OperatorTok{{-}}\DecValTok{1}\OperatorTok{;}
  \OperatorTok{\}}
  \ControlFlowTok{return}\NormalTok{ buf}\OperatorTok{;}
\OperatorTok{\}}

\DataTypeTok{double}\NormalTok{ measure\_ping\_pong\_time}\OperatorTok{(}\DataTypeTok{int}\NormalTok{ rank}\OperatorTok{,} \DataTypeTok{int}\NormalTok{ number\_amount}\OperatorTok{)} \OperatorTok{\{}
  
  \DataTypeTok{int}\NormalTok{ i}\OperatorTok{;}
  \DataTypeTok{double}\NormalTok{ t1}\OperatorTok{,}\NormalTok{ t2}\OperatorTok{;}
  \DataTypeTok{int}\OperatorTok{*}\NormalTok{ number\_buf }\OperatorTok{=}\NormalTok{ init\_buf}\OperatorTok{(}\NormalTok{number\_amount}\OperatorTok{);}

\NormalTok{  MPI\_Barrier}\OperatorTok{(}\NormalTok{MPI\_COMM\_WORLD}\OperatorTok{);}
\NormalTok{  t1 }\OperatorTok{=}\NormalTok{ MPI\_Wtime}\OperatorTok{();}
  \ControlFlowTok{for} \OperatorTok{(}\NormalTok{i }\OperatorTok{=} \DecValTok{0}\OperatorTok{;}\NormalTok{ i }\OperatorTok{\textless{}}\NormalTok{ N}\OperatorTok{;}\NormalTok{ i}\OperatorTok{++)\{}
\NormalTok{     MPI\_one\_ping\_pong}\OperatorTok{(}\NormalTok{rank}\OperatorTok{,}\NormalTok{ number\_buf}\OperatorTok{,}\NormalTok{ number\_amount}\OperatorTok{);}    
  \OperatorTok{\}} 
\NormalTok{  t2 }\OperatorTok{=}\NormalTok{ MPI\_Wtime}\OperatorTok{();}

\NormalTok{  free}\OperatorTok{(}\NormalTok{number\_buf}\OperatorTok{);}
  \ControlFlowTok{return} \OperatorTok{(}\NormalTok{t2 }\OperatorTok{{-}}\NormalTok{ t1}\OperatorTok{)/((}\DataTypeTok{double}\OperatorTok{)}\NormalTok{ N}\OperatorTok{);}
\OperatorTok{\}}

\DataTypeTok{double}\NormalTok{ compute\_thrtp}\OperatorTok{(}\DataTypeTok{double}\NormalTok{ time\_in\_sec}\OperatorTok{,} \DataTypeTok{int}\NormalTok{ number\_amount}\OperatorTok{)} \OperatorTok{\{}
  \DataTypeTok{int}\NormalTok{ buff\_size }\OperatorTok{=} \KeywordTok{sizeof}\OperatorTok{(}\DataTypeTok{int}\OperatorTok{)*}\NormalTok{number\_amount}\OperatorTok{;}
  \ControlFlowTok{return} \OperatorTok{(}\DataTypeTok{double}\OperatorTok{)} \OperatorTok{(}\DecValTok{8}\OperatorTok{*}\NormalTok{buff\_size}\OperatorTok{)/(}\FloatTok{1000000.0}\OperatorTok{*}\NormalTok{time\_in\_sec}\OperatorTok{);}
\OperatorTok{\}}

\DataTypeTok{void}\NormalTok{ export\_to\_csv}\OperatorTok{(}\DataTypeTok{int}\OperatorTok{*}\NormalTok{ buff\_size}\OperatorTok{,} \DataTypeTok{double}\OperatorTok{*}\NormalTok{ throughtputs}\OperatorTok{)} \OperatorTok{\{}
\NormalTok{  printf}\OperatorTok{(}\StringTok{"Message\_size, Throughtput}\SpecialCharTok{\textbackslash{}n}\StringTok{"}\OperatorTok{);}
  
  \DataTypeTok{int}\NormalTok{ i}\OperatorTok{;}
  \ControlFlowTok{for} \OperatorTok{(}\NormalTok{i }\OperatorTok{=} \DecValTok{0}\OperatorTok{;}\NormalTok{ i }\OperatorTok{\textless{}}\NormalTok{ TEST\_CASES\_NUM}\OperatorTok{;}\NormalTok{ i}\OperatorTok{++)} \OperatorTok{\{}
\NormalTok{    printf}\OperatorTok{(} \StringTok{"\%d, \%f}\SpecialCharTok{\textbackslash{}n}\StringTok{"}\OperatorTok{,}\NormalTok{ buff\_size}\OperatorTok{[}\NormalTok{i}\OperatorTok{],}\NormalTok{ throughtputs}\OperatorTok{[}\NormalTok{i}\OperatorTok{]);}
  \OperatorTok{\}}
\OperatorTok{\}}

\DataTypeTok{int}\NormalTok{ main }\OperatorTok{(}\DataTypeTok{int}\NormalTok{ argc}\OperatorTok{,} \DataTypeTok{char} \OperatorTok{*}\NormalTok{ argv}\OperatorTok{[])}
\OperatorTok{\{}
  \DataTypeTok{int}\NormalTok{ rank}\OperatorTok{,}\NormalTok{ size}\OperatorTok{;}
\NormalTok{  init\_mpi}\OperatorTok{(\&}\NormalTok{argc}\OperatorTok{,} \OperatorTok{\&}\NormalTok{argv}\OperatorTok{,} \OperatorTok{\&}\NormalTok{rank}\OperatorTok{,} \OperatorTok{\&}\NormalTok{size}\OperatorTok{);}
 
  \DataTypeTok{const} \DataTypeTok{int}\NormalTok{ number\_amounts}\OperatorTok{[}\NormalTok{TEST\_CASES\_NUM}\OperatorTok{]} \OperatorTok{=} \OperatorTok{\{}\DecValTok{100}\OperatorTok{,} \DecValTok{1000}\OperatorTok{,} \DecValTok{10000}\OperatorTok{,} \DecValTok{100000}\OperatorTok{,}  \DecValTok{1000000}\OperatorTok{,} \DecValTok{10000000}\OperatorTok{\};}
  \DataTypeTok{double}\OperatorTok{*}\NormalTok{ thrtps }\OperatorTok{=} \OperatorTok{(}\DataTypeTok{double}\OperatorTok{*)}\NormalTok{ malloc}\OperatorTok{(}\KeywordTok{sizeof}\OperatorTok{(}\DataTypeTok{double}\OperatorTok{)*}\NormalTok{TEST\_CASES\_NUM}\OperatorTok{);}
  \DataTypeTok{int}\OperatorTok{*}\NormalTok{ buff\_size }\OperatorTok{=} \OperatorTok{(}\DataTypeTok{int}\OperatorTok{*)}\NormalTok{ malloc}\OperatorTok{(}\KeywordTok{sizeof}\OperatorTok{(}\DataTypeTok{int}\OperatorTok{)*}\NormalTok{TEST\_CASES\_NUM}\OperatorTok{);}
  \DataTypeTok{int}\NormalTok{ i}\OperatorTok{;}  
  \ControlFlowTok{for} \OperatorTok{(}\NormalTok{i }\OperatorTok{=} \DecValTok{0}\OperatorTok{;}\NormalTok{ i }\OperatorTok{\textless{}}\NormalTok{ TEST\_CASES\_NUM}\OperatorTok{;}\NormalTok{ i}\OperatorTok{++)} \OperatorTok{\{}
    \DataTypeTok{double}\NormalTok{ snd\_rcv\_time\_sec }\OperatorTok{=}\NormalTok{ measure\_ping\_pong\_time}\OperatorTok{(}\NormalTok{rank}\OperatorTok{,}\NormalTok{ number\_amounts}\OperatorTok{[}\NormalTok{i}\OperatorTok{]);}
\NormalTok{    buff\_size}\OperatorTok{[}\NormalTok{i}\OperatorTok{]} \OperatorTok{=} \KeywordTok{sizeof}\OperatorTok{(}\DataTypeTok{int}\OperatorTok{)*}\NormalTok{number\_amounts}\OperatorTok{[}\NormalTok{i}\OperatorTok{];}
\NormalTok{        thrtps}\OperatorTok{[}\NormalTok{i}\OperatorTok{]} \OperatorTok{=}\NormalTok{ compute\_thrtp}\OperatorTok{(}\NormalTok{snd\_rcv\_time\_sec}\OperatorTok{,}\NormalTok{ number\_amounts}\OperatorTok{[}\NormalTok{i}\OperatorTok{]);} 
  \OperatorTok{\}} 
  \DataTypeTok{double}\NormalTok{ delay }\OperatorTok{=} \FloatTok{1000.0} \OperatorTok{*}\NormalTok{ measure\_ping\_pong\_time}\OperatorTok{(}\NormalTok{rank}\OperatorTok{,} \DecValTok{1}\OperatorTok{);}  

  \ControlFlowTok{if} \OperatorTok{(}\NormalTok{rank }\OperatorTok{==} \DecValTok{0}\OperatorTok{)} \OperatorTok{\{}
\NormalTok{    export\_to\_csv}\OperatorTok{(}\NormalTok{buff\_size}\OperatorTok{,}\NormalTok{ thrtps}\OperatorTok{);}
\NormalTok{        printf}\OperatorTok{(}\StringTok{"Delay: \%f[ms]}\SpecialCharTok{\textbackslash{}n}\StringTok{"}\OperatorTok{,}\NormalTok{ delay}\OperatorTok{);}
    
  \OperatorTok{\}}
\NormalTok{  free}\OperatorTok{(}\NormalTok{thrtps}\OperatorTok{);}
\NormalTok{  free}\OperatorTok{(}\NormalTok{buff\_size}\OperatorTok{);}

\NormalTok{  MPI\_Finalize}\OperatorTok{();}

  \ControlFlowTok{return} \DecValTok{0}\OperatorTok{;}
\OperatorTok{\}}
\end{Highlighting}
\end{Shaded}

Był on kompilowany bez i z flagą \texttt{-DBUFFERED} w celu zbadania
odpowiednio komunikacji niebuforowanej i buforowanej.

Wyniki pobrano zarówno dla komunikacji na 1 nodzie, jak i pomiędzy 2
node'ami (dla obu typów komunikacji). Komunikacja na 1 nodzie odbyła się
po ustawieniu \texttt{:2} przy nodzie nr 3 w pliku \texttt{allnodes}, a
komunikacja między 2 odbyła się przy ustawieniu \texttt{:1} przy nodach
nr 9 i 10.

    \hypertarget{dane-pomiarowe}{%
\subsection{Dane pomiarowe}\label{dane-pomiarowe}}

W wyniku eksperymentu uzyskano 5 plików CSV. 4 z nich odnosiły się do
przepustowości, 1 do opóźnienia. Dane pozyskano w czasie, w którym na
klastrze nie znajdowało się wielu użytkowników, co sprawdzono za pomocą
komendy \texttt{who}.

\hypertarget{przepustowoux15bux107-od-wieloux15bci-danych}{%
\subsubsection{Przepustowość od wielości
danych}\label{przepustowoux15bux107-od-wieloux15bci-danych}}

Czas przesyłu danych liczony był jako czas potrzebny na przesłanie
\texttt{x} B danych w dwie strony. Seria wielkości danych rosła
geometrycznie: każda kolejna wartość jest większa 10 razy. Każdy pomiar
powtórzono 1001 razy. Przedstawione niżej dane sa wartością oczekiwaną.

Przepustowość została zapisana w Mbit/s, wielkość danych w B. Są to 4
pliki: 1. komunikacja niebuforowana - 1 node

            \begin{tcolorbox}[breakable, size=fbox, boxrule=.5pt, pad at break*=1mm, opacityfill=0]
\begin{Verbatim}[commandchars=\\\{\}]
   Message\_size   Throughtput
0           400     20.121754
1          4000    104.291489
2         40000    225.392701
3        400000    814.356677
4       4000000   1283.516045
5      40000000   1652.449831
\end{Verbatim}
\end{tcolorbox}
        
    \begin{enumerate}
\def\labelenumi{\arabic{enumi}.}
\setcounter{enumi}{1}
\tightlist
\item
  komunikacja niebuforowana - 2 nody
\end{enumerate}

            \begin{tcolorbox}[breakable, size=fbox, boxrule=.5pt, pad at break*=1mm, opacityfill=0]
\begin{Verbatim}[commandchars=\\\{\}]
   Message\_size   Throughtput
0           400     35.568814
1          4000    189.157758
2         40000   1345.707578
3        400000   1404.841344
4       4000000   1226.464586
5      40000000   1261.470569
\end{Verbatim}
\end{tcolorbox}
        
    \begin{enumerate}
\def\labelenumi{\arabic{enumi}.}
\setcounter{enumi}{2}
\tightlist
\item
  komunikacja buforowana - 1 node
\end{enumerate}

            \begin{tcolorbox}[breakable, size=fbox, boxrule=.5pt, pad at break*=1mm, opacityfill=0]
\begin{Verbatim}[commandchars=\\\{\}]
   Message\_size   Throughtput
0           400     36.388835
1          4000    215.318647
2         40000   1320.634187
3        400000    564.011719
4       4000000   1018.657553
5      40000000   1117.138113
\end{Verbatim}
\end{tcolorbox}
        
    \begin{enumerate}
\def\labelenumi{\arabic{enumi}.}
\setcounter{enumi}{3}
\tightlist
\item
  komunikacja buforowana - 2 nody
\end{enumerate}

            \begin{tcolorbox}[breakable, size=fbox, boxrule=.5pt, pad at break*=1mm, opacityfill=0]
\begin{Verbatim}[commandchars=\\\{\}]
   Message\_size   Throughtput
0           400     37.782293
1          4000    232.925589
2         40000   1510.648549
3        400000    816.289598
4       4000000   1072.329623
5      40000000    946.367913
\end{Verbatim}
\end{tcolorbox}
        
    \hypertarget{opuxf3ux17anienie}{%
\subsubsection{Opóźnienie}\label{opuxf3ux17anienie}}

Odczytano 4 wielkości opóźnienia dla wszystkich 4 konfiguracji
eksperymentu. Została ona obliczona jako długość przesyłu w obie strony
komunikatu o wielkości 1 B. Wartość opóźnienia zapisano w ms.

            \begin{tcolorbox}[breakable, size=fbox, boxrule=.5pt, pad at break*=1mm, opacityfill=0]
\begin{Verbatim}[commandchars=\\\{\}]
       Type   Number\_of\_nodes     Delay
0    normal                 1  0.066799
1    normal                 2  0.147370
2  buffered                 1  0.066297
3  buffered                 2  0.076020
\end{Verbatim}
\end{tcolorbox}
        
    \hypertarget{wykresy}{%
\subsection{Wykresy}\label{wykresy}}

Wszystkie wykresy posiadają skalę logarytmiczną na osi x, ponieważ dane
są tam

    \hypertarget{poruxf3wnanie-przepustowoux15bci-dla-ruxf3ux17cnych-typuxf3w-komunikacji-na-1-nodzie.}{%
\subsubsection{Porównanie przepustowości dla różnych typów komunikacji
na 1
nodzie.}\label{poruxf3wnanie-przepustowoux15bci-dla-ruxf3ux17cnych-typuxf3w-komunikacji-na-1-nodzie.}}

    \begin{center}
    \adjustimage{max size={0.9\linewidth}{0.9\paperheight}}{MPI - Paweł Kruczkiewicz_files/MPI - Paweł Kruczkiewicz_16_0.png}
    \end{center}
    { \hspace*{\fill} \\}
    
    Pzepustowość dla komunikacji niebuforowanej stale rośnie. Jest to
spowodowane ciągłym zmniejszaniem narzutu związanego z samym przesyłem
danych w stosunku do całej wiadomości (procent wiadomości poświęcony na
metadane w funkcji wielkości pakietu dąży do 0). Nie udało się jednak
osiągnąć momentu ``nasycenia''. Dalsze testy były trudne do
przeprowadzenia ze względu na długość czasu przesyłu.

Buforowanie nie podlega opisanej wyżej zasadzie, ponieważ narzut jest
różny w zależności od wielkości komunikatu. Przepustowość buforowana
jest większa dla małych i średnich wielkości pakietów (do 40 KB
włącznie), jednak zdecydowanie mniejsza dla dużych pakietów. Jest to
spowodowane tym, że dla dużych komunikatów wielkość pakietu ``sama w
sobie'' stanowi bufor i dodatkowy narzut komunikacji buforowanej zwalnia
przesył danych.

\hypertarget{poruxf3wnanie-przepustowoux15bci-dla-ruxf3ux17cnych-typuxf3w-komunikacji-na-2-nodach.}{%
\subsubsection{Porównanie przepustowości dla różnych typów komunikacji
na 2
nodach.}\label{poruxf3wnanie-przepustowoux15bci-dla-ruxf3ux17cnych-typuxf3w-komunikacji-na-2-nodach.}}

    \begin{center}
    \adjustimage{max size={0.9\linewidth}{0.9\paperheight}}{MPI - Paweł Kruczkiewicz_files/MPI - Paweł Kruczkiewicz_18_0.png}
    \end{center}
    { \hspace*{\fill} \\}
    
    Wykres przepustowości do wielkości wiadomości dla 2 węzłów jest nieco
podobny jak ten dla 1 węzła, jednak można zauważyć w nim drobne różnice.

Po pierwsze - dla komunikacji niebuforowanej osiągnięto ``nasycenie''
równe ok. 1400 Mbit/s, co możemy traktować jako szczytową przepustowoś
dla tego rodzaju komunikacji.

Po drugie - różnica przepustowości dla komunikacji buforowanej nie jest
tak duża jak dla komunikacji niebuforowanej. Pozwala to stwierdzić, że
komunikacja buforowana jest bardziej istotna w środowiskach gdzie
komunikacja między procesami jest mniejsza i bufor potrzebny jest jako
równoważnik dla różnicach w szybkości przesyłu a przetwarzania danych.

\hypertarget{opuxf3ux17anienie}{%
\subsubsection{Opóźnienie}\label{opuxf3ux17anienie}}

    \begin{center}
    \adjustimage{max size={0.9\linewidth}{0.9\paperheight}}{MPI - Paweł Kruczkiewicz_files/MPI - Paweł Kruczkiewicz_20_0.png}
    \end{center}
    { \hspace*{\fill} \\}
    
    Powyższy wykres pokazuje, że buforowanie jest szczególnie korzystne dla
komunikacji między dwoma węzłami. Wartość opóźnienia nie zmienia się
znacząco dla komunikacji buforowanej, czego nie można powiedzieć o
komunikacji niebuforowanej, gdzie wartość ta jest niemal dwa razy wyższa
dla komunikacji międzywęzłowej w porównaniu z komunicją w jednym węźle.

Jest to spowodowane tym, że procesor w komunikacji buforowanej ???


    % Add a bibliography block to the postdoc
    
    
    
\end{document}
